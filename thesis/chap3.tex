\chapter{Early experiments}

\section{Microcontrollers}

My first exposure to Arduino was as an undergraduate student of electrical engineering back home in Chile. The Arduino Uno was a very powerful device, and I saw its applications to arts, when with a friend we created a rudimentary automatic tuner for guitar, that performed pitch detection and then controlled a motor to move the tuning machine on the guitar to achieve the desired tuning, with a PID controller.

I didn't use it too much, because they were relatively hard to obtain, and I was more interested in software at the time.

Fast forward to 2015, I became a graduate student at New York University's Interactive Telecommunications Program, where on my first semester I took the amazing class Introduction to Physical Computing, with one of Arduino's co-creators Tom Igoe.

While freelancing in New York, I was introduced to an Arduino off-shoot, the Teensy, which captivated me by its USB MIDI capabilities, which allowed for standalone operation without a host computer, and by its audio library, which allowed me to create interactive standalone experiences, triggering samples and applying audio effects on device.

While at MIT Media Lab, I was delighted by the newer versions of Teensy, which are even faster and more powerful, and which led me to start designing handheld samplers for field recordings.

This in turn led me to review the current NYU ITP materials for physical computing, where they currently stopped using the now classic Arduino Uno, and have incorporated 


\section{Machine learning}

Class at School of Machines by Gene Kogan and Andreas Refsgaard on 2018.

ml5.js

Machine learning for artists

Piano Die Hard with Corbin Ordel at the alt-ai conference, with Wekinator and KNN algorithm.

ml5.js is a wrapper for Tensorflow.js, NYU ITP. Browser based

Runway ML by Alejandro Matamala, Anastasis Germanidis, and Cris Valenzuela.

Casey Reas' book for GANs.

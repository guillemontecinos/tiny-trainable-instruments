\chapter{Tiny trainable instruments}

\section{Design principles}

\begin{enumerate}
  \item Cheap
  \item Privacy
\end{enumerate}

\section{Technology}

This project is based on an Arduino Nano 33 BLE Sense.

Arduino microcontroller

Arduino library KNN

TensorFlow Lite Micro

\section{Programmable / remix}

All examples included with this library were written with the aim of showing the fundamentals of how to build the instruments and different machine learning enabled manipulation of multimedia material, so that people could build on top of it and make it it their own, by changing the values of variables and adding more functionalities.

\section{Philosophy and experience}

\section{Inputs}

Enumerate sensors from the Arduino Nano 33 BLE Sense.

We are using the RGB color, proximity, gyroscope, accelerometer, and microphone sensors on the microcontroller, in order to capture the Inputs

\begin{enumerate}
  \item Color
  \item Gesture
  \item Speech
\end{enumerate}

\subsection{Color}

This approach uses the RGB color sensor from the microcontroller, with the auxiliary help from the proximity sensor, that is used to capture color information at a certain distance threshold.

The data is passed to a k-Nearest-Neighbor algorithm, programmed using the Arduino KNN library.

\subsection{Gesture}

This input uses the information from the Inertial Measurument Unit (IMU) of the microcontroller, including a gyroscope and accelerometer. It captures data after a certain threshold of movement is detected. 

The data is passed to a TensorFlow neural network, programmed using the Arduino TensorFlow Lite library, and based on the included magic$\_$wand example.

\subsection{Speech}

This input uses the information from the microphone of the microcontroller.

The data is passed to a TensorFlow neural network, programmed using the Arduino TensorFlow Lite library, and based on the included micro$\_$speech example.

\section{Outputs}

The different outputs were picked, because of their low cost, ubiquity, and possibilities of expansion and combining them.

\subsection{Buzzer}

This output creates pitched sound, by using a PWM output.

\subsection{Servo motor}

This output creates movement and through that, rhytmic sounds.

The main inspiration for this output was the emerging use of motor-activated percussive instruments, such as the Polyend Perc.

\subsection{MIDI}

We wrote functionalities to manipulate MIDI innstruments, and included examples to interface with some popular and cheap MIDI instruments, such as the Korg volca beats.

We included examples for rhythmic and melodic elements, using two very ubiquitous and inexpensive MIDI musical instruments, which are the Korg volca beats, and the Korg volca keys.

\subsection{Thermal printer}

A thermal printer is the basis for creating written and literary output, inspired by the field of computational poetry.

We used the popular Adafruit Thermal printer kit, which is documented on their website and includes a software library, distributed over GitHub and Arduino IDE, and also as a submodule on this project's TinyTrainable software library.

\section{Development}

This thesis has been developed with the invaluable help of undergrad researchers Peter Tone and Maxwell Wang.

They have cloned both repositories, the main one and the Arduino library one, and have continuously submitted pull requests with their contributions.

Peter Tone has helped with research in data structures, library writing, and we have shared back and forth code, going from experimental proofs of concepts, and has also helped with the design of the user-facing library.

Maxwell Wang has proofread our code, has ran the examples, and has helped with the writing of the documentation for self-learners and for the workshops.

We all share a Google Drive folder, where we all share notes about our research and development of the library and the educational material.

\section{Code}

This thesis is distributed as a repository, hosted on the GitHub platform, and available at github.com/montoyamoraga/tiny-trainable-instruments.

The auxiliary files, such as the LaTeX project for this document, and the auxiliary Jupyter notebooks, and documentation and tutorials are included on this repository.

The main software component of this project is the TinyTrainable library, available at github.com/montoyamoraga/TinyTrainable and also through the Arduino IDE.

The code included on this library is distributed on the folders:

\begin{enumerate}
  \item examples/
  \item src/
\end{enumerate}

\subsection{src/}

The source code for where there is a TinyTrainable.h and TinyTrainable.cpp file where we included all the basic functionality of the library. Additional subfolders include

\subsubsection{inputs/}

Base class Input and inherited classes for each one of the other inputs.

\subsubsection{outputs/}

Base class Output and inherited classes for each one of the other outputs.

\subsubsection{tensorflow/}

Auxiliary files, copied from the Arduino TensorFlow examples that we are building on top of. The files here, unless otherwise noted, are included without modifications and distributed through the Apache License included on each file's headers.

\section{Opera of the Future projects}

During the development of this thesis, I have been fortunate to collaborate on different capacities with other thesis by classmates at Opera of the Future, which has directly informed my work.

Squishies, Hannah Lienhard's master's thesis, novel squishable interfaces for musical expression. We shared discussions about low-level sound design, code reusability, sound art education, digital instruments.

Fluid Music, Charles Holbrow's PhD thesis, library design, documentation for contributors.

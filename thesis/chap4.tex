\chapter{Tiny trainable instruments}

\section{Design principles}

\begin{enumerate}
  \item Cheap
  \item Privacy
\end{enumerate}

\section{Technology}

Arduino microcontroller

Arduino library KNN

TensorFlow Lite Micro

\section{Programmable / remix}

\section{Philosophy and experience}

\section{Inputs}

Enumerate sensors from the Arduino Nano 33 BLE Sense.

\subsection{Color}

This approach uses the RGB color sensor from the microcontroller, with the auxiliary help from the proximity sensor, that is used to capture color information at a certain distance threshold.

The data is passed to a k-Nearest-Neighbor algorithm, programmed using the Arduino KNN library.

\subsection{Gesture}

This input uses the information from the Inertial Measurument Unit (IMU) of the microcontroller, including a gyroscope and accelerometer. It captures data after a certain threshold of movement is detected.

The data is passed to a TensorFlow neural network, programmed using the Arduino TensorFlow Lite library, and based on the included magic_wand example.

\subsection{Speech}

This input uses the information from the microphone of the microcontroller.

The data is passed to a TensorFlow neural network, programmed using the Arduino TensorFlow Lite library, and based on the included micro_speech example.

\section{Outputs}

The different outputs were picked, because of their low cost, ubiquity, and possibilities of expansion and combining them.

\subsection{Buzzer}

This output creates pitched sound, by using a PWM output.

\subsection{Servo motor}

This output creates movement and through that, rhytmic sounds.

\subsection{MIDI}

We wrote functionalities to manipulate MIDI innstruments, and included examples to interface with some popular and cheap MIDI instruments, such as the Korg volca beats.

\subsection{Thermal printer}

A thermal printer is the basis for creating written and literary output, inspired by the field of computational poetry.

\section{Development}

This thesis has been developed with the invaluable help of undergrad researchers Peter Tone and Maxwell Wang.

They have cloned both repositories, the main one and the Arduino library one, and have continuously submitted pull requests with their contributions.

Peter Tone has helped with research in data structures, library writing, and we have shared back and forth code, going from experimental proofs of concepts, and has also helped with the design of the user-facing library.

Maxwell Wang has proofread our code, has ran the examples, and has helped with the writing of the documentation for self-learners and for the workshops.

We all share a Google Drive folder, where we all share notes about our research and development of the library and the educational material.

\section{Code}

This thesis is distributed as a repository, hosted on the GitHub platform, and available at https://github.com/montoyamoraga/tiny-trainable-instruments.



The auxiliary files, such as the LaTeX project for this document, and the auxiliary Jupyter notebooks, and documentation and tutorials are on the repository .


The code of this thesis is written in C++, and is packaged as an Arduino library.

It is distributed through the repository https://github.com/montoyamoraga/TinyTrainable and also available through the Arduino IDE.


\section{Opera of the Future projects}

During the development of this thesis, I have been fortunate to collaborate on different capacities with other thesis by classmates at Opera of the Future, which has directly informed my work.

Squishies, Hannah Lienhard's master's thesis, novel squishable interfaces for musical expression. We shared discussions about low-level sound design, code reusability, sound art education, digital instruments.

Fluid Music, Charles Holbrow's PhD thesis, library design, documentation for contributors.
